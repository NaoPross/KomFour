% !TeX program = xelatex
% !TeX encoding = utf8
% !TeX root = komfour_zf.tex

%% TODO: publish to CTAN
\documentclass[]{tex/hsrzf}

%%%%%%%%%%%%%%%%%%%%%%%%%%%%%%%%%%%%%%%%%%%%%%%%%%%
% Packages

%% TODO: publish to CTAN
\usepackage{tex/hsrstud}

%% Language configuration
\usepackage{polyglossia}
\setdefaultlanguage[variant=swiss]{german}

%% Math
\usepackage{amsmath}
\usepackage{amsthm}

%% Layout
\usepackage{enumitem}


%% License configuration
\usepackage[
  type={CC},
  modifier={by-nc-sa},
  version={4.0},
  lang={german},
]{doclicense}

%%%%%%%%%%%%%%%%%%%%%%%%%%%%%%%%%%%%%%%%%%%%%%%%%%%
% Metadata

\course{Elektrotechnik}
\module{KomFour}
\semester{Fr\"uhlingssemester 2020}

\authoremail{npross@hsr.ch}
\author{Naoki Pross -- \texttt{\theauthoremail}}

\title{\texttt{\themodule} Zusammenfassung}
\date{\thesemester}

%%%%%%%%%%%%%%%%%%%%%%%%%%%%%%%%%%%%%%%%%%%%%%%%%%%
% Macros and settings

%% Equal by definition
\newcommand\defeq{\overset{\mathrm{def.}}{=}}

%% number sets
\newcommand\Nset{\mathbb{N}}
\newcommand\Rset{\mathbb{R}}
\newcommand\Cset{\mathbb{C}}

%% Complex operators
\DeclareMathOperator\cjs{cjs}
\newcommand\cjsl[1]{\cos #1 + j\sin #1}

\newcommand\ej[1]{e^{j#1}}

\newcommand\conj[1]{\overline{#1}}
\newcommand\len[1]{\lvert#1\rvert}

\renewcommand\Re{\operatorname{Re}}
\renewcommand\Im{\operatorname{Im}}

%% Theorems
\newtheoremstyle{komfourzf} % name of the style to be used
  {\topsep}
  {\topsep}
  {}
  {0pt}
  {\bfseries}
  {.}
  { }
  { }

\theoremstyle{komfourzf}
\newtheorem{theorem}{Satz}

\setlist[description]{%
  align=right, labelwidth=2cm, leftmargin=!, %
  format={\normalfont\slshape}}

\setlist[itemize]{%
  align=right, labelwidth=5mm, leftmargin=!}

%%%%%%%%%%%%%%%%%%%%%%%%%%%%%%%%%%%%%%%%%%%%%%%%%%%
% Document

\begin{document}

\maketitle
\tableofcontents

\section{Komplexe Zahlen}

\begin{theorem}[Komplexe Einheit]
\(
  j \defeq +\sqrt{-1} \iff j^2 = -1
\)
\end{theorem}

\begin{theorem}[Summe] Seien \(a, b \in \Cset\),
  \(a = a_1 + ja_2, a_1,a_2 \in \Rset\) und \"ahnlich f\"ur \(b\)
\[
  a \oplus b =  (a_1 + b_1) + j (a_2 + b_2)
\]
\end{theorem}

\begin{theorem}[Multiplikation] Seien \(a, b \in \Cset\)
  \(\arg a = \phi, \arg b = \theta\)
  \begin{description}
    \item[Kartesich] \(a \odot b = (a_1 b_1 - a_2 b_2) + j (a_1 b_2 + a_2 b_1)\)
    \item[Polar] \(a\odot b = |a|\cdot|b|\exp{j(\phi + \theta)}\)
  \end{description} 
\end{theorem}

\begin{theorem}[Division] Seien \(a, b \in \Cset\)
  mit \(\arg a = \phi, \arg b = \theta\),
  dann \(a / b = |a|/|b|\exp{j(\phi - \theta)}\)
\end{theorem}

\begin{theorem}[Potenzen]~
  \begin{itemize}
    \item F\"ur \(n \in \Nset\) gilt
      \(\cjs(x)^n = \cjs(nx) \iff \left(\ej{x}\right)^n = \ej{nx}\)
    \item 
  \end{itemize}
\end{theorem}

\begin{theorem}[Wurzeln] Sei \(\Cset \ni z = r\ej{\phi}\).
  \(z\) hat genau \(n\) verschiedene \(n\)-te Wurzeln
  (\(n \in \Nset\))
  \[
    w_{k+1} = \sqrt[n]{r}\exp \frac{j(\phi + 2k\pi)}{n}
      \qquad k = 0,1,\ldots,n-1
  \]
  Beachtung! Allgemein \(a,b \in \Cset: \sqrt[n]{ab} \neq \sqrt[n]{a}\sqrt[n]{b}\)
\end{theorem}

\begin{theorem}[Polynome in \(\Cset\)]~ %
  \begin{itemize}
    \item Jedes komplexe Polynom vom Grad \(\geq 1\) hat mindestens eine Nullstelle.

    \item Ein komplexes Polynom \(p(z) = a_n z^n + \cdots + a_1 z + a_0\) vom 
      Grad \(n\) zerf\"allt in \(\Cset\) in lauter lineare Faktoren, wobei \(z_k 
      \in \Cset\) als Nullstellen von \(p(z)\) nicht unbedingt verschieden sein 
      m\"ussen.

    \item Ein komplexes Polynom \(p(z)\) vom Grad \(n\) hat in \(\Cset\) genau 
      \(n\) (verschiedene) Nullstellen, wenn diese mit ihrer Vielfachheit 
      gez\"ahlt werden.
  \end{itemize}
\end{theorem}

\begin{theorem}[Polynome mit reellen Koeffizienten]~ %
  \begin{itemize}
    \item F\"ur Polynome mit reellen Koeffizienten \(p(z)\) treten nicht-reelle 
      Nullstellen nur als \emph{konjugiert-komplexe} Paare \(w, \conj{w}\).
      In der komplexen Linearfaktor-Zerlegung von \(p(z)\) k\"onnen dan wei 
      Faktoren \((z-z_0)\) und \((z-\conj{z_0})\) jeweils zu einem 
      quadratischen Faktor \[
        z^2 - 2 \Re(z_0) z + \len{z}^2
      \] mit \emph{reellen} Koeffizienten zusammengefasst werden.

    \item Ein Polynom mit reellen Koeffizienten von \emph{ungeraden} Grad hat
      mindestens eine \emph{reelle} Nullstelle.

    \item Alle Nullstellen des Polynoms \(p(z) = a_n z^n + \cdot + a_1 z + a_0\)
      liegen in der Gauss'schen Zahlenebene in einer Kreisscheibe um der 
      Ursprung mit Radius \[
        R = \sum_{k=0}^n \left\lvert\frac{a_k}{a_n}\right\rvert
      \]

    %% TODO: kubische Gleichung

    \item F\"ur allgemeine Gleichungen vom Grad 5 und gr\"osser existieren 
      prinzipiell \emph{keine} nur aus den 4 Grundoperationen und Wurzeln
      zusammengesetzten L\"osungsformeln.
  \end{itemize}
\end{theorem}

\section{Komplexwertige Funktionen}
\section{Fourierreihen}
\section{Spektren}
\section{Diskrete Fouriertransformation}

\section{Lizenz}
\doclicenseThis

\end{document}
% vim: set et ts=2 sw=2 spelllang=de spell linebreak :
