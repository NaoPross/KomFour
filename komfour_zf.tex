% !TeX program = xelatex
% !TeX encoding = utf8
% !TeX root = komfour_zf.tex

%% TODO: publish to CTAN
\documentclass[twocolumn, small]{tex/hsrzf}

%%%%%%%%%%%%%%%%%%%%%%%%%%%%%%%%%%%%%%%%%%%%%%%%%%%
% Packages

%% TODO: publish to CTAN
\usepackage{tex/hsrstud}

%% Language configuration
\usepackage{polyglossia}
\setdefaultlanguage[variant=uk]{english}

%% Math
\usepackage{amsmath}
\usepackage{amsthm}

%% Layout
\usepackage{enumitem}


%% License configuration
\usepackage[
  type={CC},
  modifier={by-nc-sa},
  version={4.0},
  lang={english},
]{doclicense}

%%%%%%%%%%%%%%%%%%%%%%%%%%%%%%%%%%%%%%%%%%%%%%%%%%%
% Metadata

\course{Elektrotechnik}
\module{KomFour}
\semester{Fr\"uhlingssemester 2020}

\authoremail{npross@hsr.ch}
\author{Naoki Pross -- \texttt{\theauthoremail}}

\title{\texttt{\themodule} Zusammenfassung}
\date{\thesemester}

%%%%%%%%%%%%%%%%%%%%%%%%%%%%%%%%%%%%%%%%%%%%%%%%%%%
% Macros and settings

%% Equal by definition
\newcommand\defeq{\overset{\mathrm{def.}}{=}}

%% number sets
\newcommand\Nset{\mathbb{N}}
\newcommand\Rset{\mathbb{R}}
\newcommand\Cset{\mathbb{C}}

%% Complex operators
\DeclareMathOperator\cjs{cjs}
\newcommand\cjsl[1]{\cos #1 + j\sin #1}

\newcommand\ej[1]{e^{j#1}}

\newcommand\conj[1]{\overline{#1}}
\newcommand\len[1]{\lvert#1\rvert}

\renewcommand\Re{\operatorname{Re}}
\renewcommand\Im{\operatorname{Im}}

%% Theorems
\newtheoremstyle{komfourzf} % name of the style to be used
  {\topsep}
  {\topsep}
  {}
  {0pt}
  {\bfseries}
  {.}
  { }
  { }

\theoremstyle{komfourzf}
\newtheorem{theorem}{Theorem}
\newtheorem{definition}{Definition}
\newtheorem{lemma}{Lemma}

\setlist[description]{%
  align=right, labelwidth=2cm, leftmargin=!, %
  format={\normalfont\itshape}}

\setlist[itemize]{%
  align=right, labelwidth=5mm, leftmargin=!}

%%%%%%%%%%%%%%%%%%%%%%%%%%%%%%%%%%%%%%%%%%%%%%%%%%%
% Document

\begin{document}

\section{Complex Numbers}

\begin{definition}[Complex Unit and Zero]
  \[
    j \defeq +\sqrt{-1} \iff j^2 = -1
  \]
  \[
    1 = (1,0) \quad 0 = (0,0) \quad j = (0,1)
  \]
\end{definition}

\begin{definition}[Negation and Sum] Let \(z, w \in \Cset\)
\[
  -z = (-z_1, -z_2) \quad
  z \oplus w =  (z_1 + w_1, z_2 + w_2)
\]
\end{definition}

\begin{lemma}
  The complex numbers form an additive group. Let \(z, w, v \in \Cset\), we have
  \begin{description}[leftmargin=3cm]
    \item[Identity] \(z + 0 = z\)
    \item[Commutativity] \(z + w = w + z\)
    \item[Associativity] \(z + (w + v) = (z + w) + v\)
    \item[Inverse property] \(z + (-z) = (-z) + z = 0\)
  \end{description}
\end{lemma}

\begin{definition}[Multiplication] Let \(z, w \in \Cset\)
  \[
    (a,b) \odot (c,d) = (ac - bd, ad + bc)
  \]
\end{definition}

\begin{lemma} The complex numbers form a commutative ring. Let \(z,w,v \in\Cset\)
  \begin{description}[leftmargin=3cm]
    \item[Identity] \(1\cdot z = z\)
    \item[Commutativity] \(z \cdot w = w \cdot z\)
    \item[Associativity] \(z (w v) = (z w) v\)
    \item[Distributivity] \(z (w + v) = zw + zv\)
  \end{description}
\end{lemma}

\begin{definition}[Real and imaginary part and conjugation]
  Let \(z = a + jb\). The \emph{real} part of \(z\) is \(\Re(z) = a\), similarly the \emph{imaginary} part is \(\Im(z) = b\). We can thus define the \emph{complex conjugate} \(\conj{z}\).
  \[
    z = \Re(z) + j\Im(z)
    \quad
    \conj{z} = \Re(z) - j\Im(z)
  \]
\end{definition}

\begin{definition}[Absolute value]
  If \(z = a + jb\) we define the \emph{abolute value} \(\len{z} = \sqrt{a^2 + b^2}\)
\end{definition}

\begin{lemma}[Properties of absolute value]
  Let \(z,w\in\Cset\). We have \(\len{z} \in\Rset^+_0\), \(z\conj{z} = \len{z}^2\) and as a consequence \(\len{zw} = \len{z}\cdot\len{w}\) and \(\len{\conj{z}} = \len{z}\). In additon we have the inequalities
  \begin{align*}
    -\len{z} \leq &\Re(z) \leq \len{z} &
    \len{z} &\leq \len{\Re(z)} + \len{\Im(z)} \\
    -\len{z} \leq &\Im(z) \leq \len{z} &
    \len{z + w} &\leq \len{z} + \len{w}
  \end{align*}
  The last one is the \emph{triangle inequality}.
\end{lemma}

\begin{definition}[Reciprocal and quotients]
  If \(z\) is a non-zero complex number we define the \emph{reciprocal} \(z^{-1}\) of \(z\) to be \(z^{-1} = \len{z}^{-2}\conj{z}\). If \(z = 0\) the reciprocal \(0^{-1}\) is left undefined.
  It is now possible to define \(z/w = zw^{-1}\) with \(z,w \in\Cset\) and \(w \neq 0\).
\end{definition}

\begin{lemma}[More properties of conjugation]
  Let \(z,w \in\Cset\).
  \(\conj{z} = z\) iff \(z \in \Rset\) and \(\conj{z} = \conj{w}\) iff \(z = w\).
  Furthermore:
  \begin{align*}
    \conj{\conj{z}} &= z &
    \conj{z \pm  w} &= \conj{z} \pm \conj{w} &
    \Re(z) &= (z + \conj{z})/2 \\
    \conj{z\cdot w} &= \conj{z}\cdot\conj{w} &
    \conj{z/w} &= \conj{z}/\conj{w} &
    \Im(z) &= (z - \conj{z})/2j
  \end{align*}
\end{lemma}

\begin{definition}[Polar notation]
  An alternative rapresentation of a complex number \(z = a + jb\) is its \emph{polar form} \(z = r \angle \phi\), where \(r = \len{z}\) and \(\phi = \arg{z} = r(\cos\phi + j\sin\phi)\).
  \begin{align*}
    a &= r\cos\phi &
    b &= r \sin\phi &
    r &= \sqrt{z\conj{z}}
  \end{align*}
  For \(a = 0\) we define \(\phi = \lim_{a\to 0} \arctan(b/a)\) and otherwise
  \begin{align*}
    \phi = \arg(z)
    &= \begin{cases}
      \arctan(b/a) & a > 0 \\
      \arctan(b/a) + \pi & a < 0
    \end{cases} \\
    &= \begin{cases}
      \arccos(a/r) & b \geq 0 \\
      -\arccos(b/r) & b < 0 \\
    \end{cases}
  \end{align*}
  Another variant of this notation is
  \[
    z = r\cjs\phi = r(\cos\phi + j\sin\phi)
  \]
\end{definition}

\begin{lemma}[Arithmetic in polar notation]
  Let \(z,w\in\Cset\) then the product \(zw\) has
  \[
    \len{zw} = \len{z}\cdot\len{w} \quad
    \arg(zw) = \arg z + \arg w
  \]
  Similarly the quotient \(z/w\) follows
  \[
    \len{z/w} = \len{z}/\len{w} \quad
    \arg(z/w) = \arg z - \arg w
  \]
\end{lemma}

\begin{theorem}[De Moivre's formula]
  Let \(n \in\Nset\)
  \[
    \left(\cos\phi + j\sin\phi\right)^n = \cos(n\phi) + j\sin(n\phi)
  \]
  As a consequence with the binomial formula
  \((a + b)^n = \sum_{k=0}^n \binom{n}{k} a^{n-k} b^n\) we have
  \[
    \sin(n\phi) = 
  \]
\end{theorem}

\section{Complex valued functions}

\section{License}
\doclicenseThis

\end{document}
% vim: set et ts=2 sw=2 spelllang=us spell linebreak :
