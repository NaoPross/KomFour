% !TeX program = xelatex
% !TeX encoding = utf8
% !TeX root = komfour_zf.tex

%% TODO: publish to CTAN
\documentclass[twocolumn, margin=small]{tex/hsrzf}

%%%%%%%%%%%%%%%%%%%%%%%%%%%%%%%%%%%%%%%%%%%%%%%%%%%
% Packages

%% TODO: publish to CTAN
\usepackage{tex/hsrstud}

%% Language configuration
\usepackage{polyglossia}
\setdefaultlanguage[variant=uk]{english}

%% Math
\usepackage{amsmath}
\usepackage{amsthm}

%% Layout
\usepackage{enumitem}


%% License configuration
\usepackage[
  type={CC},
  modifier={by-nc-sa},
  version={4.0},
  lang={english},
]{doclicense}

%%%%%%%%%%%%%%%%%%%%%%%%%%%%%%%%%%%%%%%%%%%%%%%%%%%
% Metadata

\course{Electrical Engineering}
\module{KomFour}
\semester{Spring Semester 2020}

\authoremail{npross@hsr.ch}
\author{Naoki Pross -- \texttt{\theauthoremail}}

\title{Cheat sheets for \texttt{\themodule}}
\date{\thesemester}

%%%%%%%%%%%%%%%%%%%%%%%%%%%%%%%%%%%%%%%%%%%%%%%%%%%
% Macros and settings

%% Equal by definition
\newcommand\defeq{\overset{\mathrm{def.}}{=}}

%% number sets
\newcommand\Nset{\mathbb{N}}
\newcommand\Zset{\mathbb{Z}}
\newcommand\Qset{\mathbb{Q}}
\newcommand\Rset{\mathbb{R}}
\newcommand\Cset{\mathbb{C}}

%% Missing operators
\DeclareMathOperator\sgn{sgn}

%% Complex operators
\DeclareMathOperator\cjs{cjs}
\newcommand\cjsl[1]{\cos #1 + j\sin #1}

\newcommand\ej[1]{e^{j#1}}

\newcommand\conj[1]{\overline{#1}}
\newcommand\len[1]{\lvert#1\rvert}

\renewcommand\Re{\operatorname{Re}}
\renewcommand\Im{\operatorname{Im}}

%% Theorems
\newtheoremstyle{komfourzf} % name of the style to be used
  {\topsep}
  {\topsep}
  {}
  {0pt}
  {\bfseries}
  {.}
  { }
  { }

\theoremstyle{komfourzf}
\newtheorem{theorem}{Theorem}
\newtheorem{definition}{Definition}
\newtheorem{lemma}{Lemma}

\setlist[description]{%
  align=right, labelwidth=2cm, leftmargin=!, %
  format={\normalfont\itshape}}

\setlist[itemize]{%
  align=right, labelwidth=5mm, leftmargin=!}

%%%%%%%%%%%%%%%%%%%%%%%%%%%%%%%%%%%%%%%%%%%%%%%%%%%
% Document

\begin{document}

\section{Complex Numbers}

\begin{definition}[Complex Unit and Zero]
  \[
    j \defeq +\sqrt{-1} \iff j^2 = -1
  \]
  \[
    1 = (1,0) \quad 0 = (0,0) \quad j = (0,1)
  \]
\end{definition}

\begin{definition}[Negation and Sum] Let \(z, w \in \Cset\)
\[
  -z = (-z_1, -z_2) \quad
  z \oplus w =  (z_1 + w_1, z_2 + w_2)
\]
\end{definition}

\begin{lemma}
  The complex numbers form an additive group. Let \(z, w, v \in \Cset\), we have
  \begin{description}[leftmargin=3cm]
    \item[Identity] \(z + 0 = z\)
    \item[Commutativity] \(z + w = w + z\)
    \item[Associativity] \(z + (w + v) = (z + w) + v\)
    \item[Inverse property] \(z + (-z) = (-z) + z = 0\)
  \end{description}
\end{lemma}

\begin{definition}[Multiplication] Let \(z, w \in \Cset\)
  \[
    (a,b) \odot (c,d) = (ac - bd, ad + bc)
  \]
\end{definition}

\begin{lemma} The complex numbers form a commutative ring. Let \(z,w,v \in\Cset\)
  \begin{description}[leftmargin=3cm]
    \item[Identity] \(1\cdot z = z\)
    \item[Commutativity] \(z \cdot w = w \cdot z\)
    \item[Associativity] \(z (w v) = (z w) v\)
    \item[Distributivity] \(z (w + v) = zw + zv\)
  \end{description}
\end{lemma}

\begin{definition}[Real and imaginary part and conjugation]
  Let \(z = a + jb\). The \emph{real} part of \(z\) is \(\Re(z) = a\), similarly the \emph{imaginary} part is \(\Im(z) = b\). We can thus define the \emph{complex conjugate} \(\conj{z}\) of \(z\) to be
  \[
    z = \Re(z) + j\Im(z)
    \quad
    \conj{z} = \Re(z) - j\Im(z)
  \]
\end{definition}

\begin{definition}[Absolute value]
  If \(z = a + jb\) we define the \emph{absolute value} \(\len{z} = \sqrt{a^2 + b^2}\)
\end{definition}

\begin{lemma}[Properties of absolute value]
  Let \(z,w\in\Cset\). We have \(z\conj{z} = \len{z}^2\) and as a consequence \(\len{zw} = \len{z}\cdot\len{w}\) and \(\len{\conj{z}} = \len{z}\). In addition we have the inequalities
  \begin{align*}
    -\len{z} \leq &\Re(z) \leq \len{z} &
    \len{z} &\leq \len{\Re(z)} + \len{\Im(z)} \\
    -\len{z} \leq &\Im(z) \leq \len{z} &
    \len{z + w} &\leq \len{z} + \len{w}
  \end{align*}
  The last one is the \emph{triangle inequality}. Notice that \(\len{z} \in\Rset^+_0\).
\end{lemma}

\begin{definition}[Reciprocal and quotients]
  If \(z\) is a non-zero complex number we define the \emph{reciprocal} \(z^{-1}\) of \(z\) to be \(z^{-1} = \len{z}^{-2}\conj{z}\). If \(z = 0\) the reciprocal \(0^{-1}\) is left undefined.
  It is now possible to define \(z/w = zw^{-1}\) with \(z,w \in\Cset\) and \(w \neq 0\).
\end{definition}

\begin{lemma}[Properties of conjugation]
  Let \(z,w \in\Cset\).
  \(\conj{z} = z\) iff \(z \in \Rset\) and \(\conj{z} = \conj{w}\) iff \(z = w\).
  Furthermore:
  \begin{align*}
    \conj{\conj{z}} &= z &
    \conj{z \pm  w} &= \conj{z} \pm \conj{w} &
    \Re(z) &= (z + \conj{z})/2 \\
    \conj{z\cdot w} &= \conj{z}\cdot\conj{w} &
    \conj{z/w} &= \conj{z}/\conj{w} &
    \Im(z) &= (z - \conj{z})/2j
  \end{align*}
\end{lemma}

\begin{definition}[Argument and polar notation]
  An alternative representation of a complex number \(z = a + jb\) is its \emph{polar form} \(z = r \angle \phi\), where \(r = \len{z}\) and \(\phi = \arg{z}\).
  \begin{align*}
    a &= r\cos\phi &
    b &= r \sin\phi &
    r &= \sqrt{z\conj{z}}
  \end{align*}
  For \(a = 0\) we define \(\phi = \lim_{a\to 0} \arctan(b/a) = \pm\pi/2\) and otherwise
  \begin{align*}
    \phi = \arg(z)
    &= \begin{cases}
      \arctan(b/a) & a > 0 \\
      \arctan(b/a) + \pi & a < 0
    \end{cases} \\
    &= \begin{cases}
      \arccos(a/r) & b \geq 0 \\
      -\arccos(b/r) & b < 0 \\
    \end{cases}
  \end{align*}
  Another variant of this notation is
  \[
    z = r\cjs\phi = r(\cos\phi + j\sin\phi)
  \]
\end{definition}

\begin{lemma}[Arithmetic in polar notation]
  Let \(z,w\in\Cset\) then the product \(zw\) has
  \[
    \len{zw} = \len{z}\cdot\len{w} \quad
    \arg(zw) = \arg z + \arg w
  \]
  Similarly the quotient \(z/w\) follows
  \[
    \len{z/w} = \len{z}/\len{w} \quad
    \arg(z/w) = \arg z - \arg w
  \]
  Lastly from the product we see that for \(k \in \Nset\)
  \[
    \len{z^k} = \len{z}^k \quad
    \arg{z^k} = k \arg{z}
  \]
\end{lemma}

\begin{theorem}[De Moivre's formula]
  Let \(n \in\Nset\)
  \[
    \left(\cos\phi + j\sin\phi\right)^n = \cos(n\phi) + j\sin(n\phi)
  \]
  As a consequence with the binomial formula
  \((a + b)^n = \sum_{k=0}^n \binom{n}{k} a^{n-k} b^n\), recalling that \(\binom{n}{k} = n!/(k!(n-k)!)\) (Pascal's triangle), we have
  \begin{align*}
    \sin(nx)&=\sum _{k=0}^{n}{\binom {n}{k}}(\cos x)^{k}\,(\sin x)^{n-k}\,\sin {\frac {(n-k)\pi }{2}}\\
    \cos(nx)&=\sum _{k=0}^{n}{\binom {n}{k}}(\cos x)^{k}\,(\sin x)^{n-k}\,\cos {\frac {(n-k)\pi }{2}}
  \end{align*}
\end{theorem}

\section{Complex valued functions}

\begin{definition}[Function in \(\Cset\)]
  Let \(f: \mathbb{D} \to \mathbb{W}\) with both \(\mathbb{D}, \mathbb{W} \subseteq \Cset\) 
  that maps \(z = (a + jb) \mapsto w = (u + jv)\),
  then \(u = \Re f(z)\) and \(v = \Im f(z)\).
  If \(f\) is a bijection with inverse \(f^{-1}\), then \(a = \Re f^{-1}(w), b = \Im f^{-1}(w)\).
\end{definition}

\begin{definition}[Differentiation in \(\Cset\)]
  Let \(f\) be a function of \(z\) and \(h \in \Cset\). We have the limit
  \[
    \lim_{\len{h} \to 0} \frac{f(z_0 + h) - f(z_0)}{h} = f'(z_0)
  \]
  to define the \emph{derivative} of \(f\) at the point \(z_0\).
\end{definition}

\begin{lemma}[Local dilation and rotation]
  Let \(f\) be a differentiable function in \(\Cset\).
  If \(f'(z) \neq 0\) everywhere, then \(f\) is a conformal map (i.e. preserves angles) with local dilation of \(\len{f'(z)}\) and rotation of \(\arg f'(z)\)
\end{lemma}

\begin{definition}[Linear function]

\end{definition}

\begin{definition}[Monomial and \(n\)-th root]
  Let \(w = z^n\) be a monomial of degree \(n\in\Nset\). Using the polar notation we see that
  \((r\angle \phi)^n = r^n \angle (n\phi)\). Because \(r\angle\phi = r\angle(\phi+2\pi)\) there cannot be a bijection between \(w\) and \(z\), if we want to define an inverse function \(z = \sqrt[n]{w}\) we get many values with the form
  \[
    z_k = \sqrt[n]{r}\angle(\phi + k2\pi)/n \qquad 0 \leq k < n
  \]
  This fact implies that in general for \(z,u \in\Cset\) \(\sqrt[n]{zu} \neq \sqrt[n]{z}\sqrt[n]{u}\), as the relationship holds only for \emph{some} values of \(\sqrt[n]{z} \text{ and } \sqrt[n]{u}\).
\end{definition}

\begin{theorem}[Roots of a polynomial]
  Every complex polynomial of degree \(n\) has always \(n\) roots in \(\Cset\).
\end{theorem}

\begin{theorem}
  Every complex polynomial of degree \(n\) with coefficients can be \emph{uniquely} rewritten in term of its roots.
  \[
    P(z) = \sum_{k=0}^n a_k z^k = a_n \prod_{k=0}^{n} (z - z_k)
  \]
\end{theorem}

\begin{theorem}[Polynomal with real coefficients]
  The roots of a polynomial with real coefficients of degree \(n\), always come in conjugate complex pairs of \(r\) and \(\conj{r}\). That is because
  \[
    (z - r)(z - \conj{r}) = z^2 - 2\Re(r)z + \len{z}^2
  \]
\end{theorem}

\begin{lemma}
  From the previous theorem follows that a polynomial with real coefficients of \emph{odd} degree, has \emph{always} at least one real solution because \(r \in\Rset \iff r = \conj{r}\).
\end{lemma}

\begin{theorem}
  All roots of a polynomial \(p(z) = \sum_{k=0}^n a_k z^k\) are inside of the open disk centered at the origin of radius \(\sum_{k=0}^n \len{a_k / a_n}\).
\end{theorem}

\begin{theorem}[Cardano's cubic formula]
  % TODO
\end{theorem}

\begin{definition}[Exponential]
  If \(z\) is a complex number we define the exponential function \(e^z\) by its convergent power series
  \[
    e^z = \sum_{n=0}^\infty \frac{z^n}{n!}
  \]
\end{definition}

\begin{theorem}[Euler's formula]
  By setting the argument of the exponential function to \(jt\) for some \(t \in\Rset\) we can reorder the power series to be a sum of the power series of \(j\sin\) and \(\cos\), and thus define
  \[
    e^{jt} = \cos t + j\sin t = \cjs t = 1\angle t
  \]
\end{theorem}

\begin{lemma}[Rules for exponents]
  Let \(a,b \in\Cset\) and \(k\in\Zset\), we can show that
  \[
    e^a e^b = e^{a+b} \quad
    e^a / e^b = e^{a-b} \quad
    \left(e^a \right)^k = e^{ak}
  \]
\end{lemma}

\begin{definition}[Trigonometric functions]
  When \(z\) is a complex number we define
  \begin{align*}
    \cos z &= \frac{e^{jz} + e^{-jz}}{2} &
    \sin z &= \frac{e^{jz} - e^{-jz}}{2j}
  \end{align*}
  like the (real) hyperbolic trigonometric functions
  \begin{align*}
    \cosh z &= \left( e^z + e^{-z} \right)/2 &
    \sinh z &= \left( e^z - e^{-z} \right)/2
  \end{align*}
  Notice that the sinus function is point symmetric to \(\pi/2\), because \(\sin(\pi/2 - z) = \sin(\pi/2 + z)\).
\end{definition}

\begin{lemma}[Some trigonometric identities] Let \(x,a,b \in\Rset\) and \(\alpha,\beta \in\Cset\)
  \begin{align*}
    \sinh(jx) &= j\sin(x) \qquad \cosh(jx) = \cos(x) \\
    \sin(a + jb) &= \sin(a)\cosh(b) + j\cos(a)\sinh(b) \\
    \cos(a + jb) &= \cos(a)\cosh(b) + j\sin(a)\sinh(b) \\
    2\sin(\alpha)\sin(\beta) &= \cos(\alpha - \beta) - \cos(\alpha + \beta) \\
    2\sin(\alpha)\cos(\beta) &= \sin(\alpha - \beta) + \sin(\alpha + \beta)
  \end{align*}
\end{lemma}

\begin{lemma}[Superposition of sinuses]
  Let \(s(t) = A\sin(\omega t + \varphi)\) be a sinusoidal wave.
  We can rewrite \(s\) in complex form with
  \[
    s(t) = \Im\left(Ae^{j(\omega t + \varphi)}\right) = \Im
      Ae^{j\varphi}\cdot e^{j\omega t}
  \]
  If we now wish to sum \(N\) sinusoids with the same frequency \(\omega\), we can set
  % TODO: Satz 18
\end{lemma}

\begin{definition}[Logarithm]
  Because \(w = e^z\) defined from \(\Cset \to \Cset\) is not a bijection (\(e^{z + 2\pi j} = e^z\)), unless we restrict the imaginary part of the domain to \((\pi, \pi]\), we get only an equivalence relationship because
  \[
    \ln\left[\len{w} e^{j(\phi + k2\pi)}\right] = \ln\len{w} + j(\phi + k2\pi)
  \]
  where \(k \in\Zset\). Similarly for \(z,w\in\Cset\)
  \begin{align*}
    \ln(w) &\equiv z &\pmod{2\pi j} \\
    \ln(w^k) &\equiv k\ln(w) &\pmod{2\pi j} \\
    \ln(zw) &\equiv \ln(z) + \ln(w) &\pmod{2\pi j} \\
    \ln(z/w) &\equiv \ln(z) - \ln(w) &\pmod{2\pi j}
  \end{align*}
\end{definition}

\begin{lemma}[General exponentiation]
  So far we have only exponentiation for an exponent \(k\in\Zset\), by adding \(m \in\Nset\) we can define the quotient \(k/m \in\Qset\) that together with \(z\in\Cset\) gives
  \begin{align*}
    z^{k/m} &= e^{\ln(z) k/m} \\
    &= \exp\big((\ln\len{z} + j(\arg z + 2\pi n)) k/m\big) \\
    &= \exp\big(\ln\len{z}\cdot k/m)\exp((\arg z + 2\pi n)jk/m\big) \\
    &= \len{z}^{k/m}\exp\big((\arg z + 2\pi n)jk/m\big)= \sqrt[m]{z^k}
  \end{align*}
  like in the reals, except that we have \(m\) values because of the \(m\)-th root. If we let \(w \in\Cset\) the expression \(z^w\) cannot be equal to an unique value because
  \begin{align*}
    z^w = e^{w \ln z} &= \exp\big( w (\ln\len{z} + j \arg{z} + 2\pi nj)\big) \\
    &= e^{w(\ln\len{z} + j\arg z)} e^{w2\pi nj}
  \end{align*}
  instead it is said to be \emph{multivalued}. This means that there are no general exponentiation rules.
\end{lemma}

\section{Fourier Series}
\begin{definition}[Real trigonometric polynomial]
  Let \(\omega = 2\pi/T \in\Rset\) and \(A_n, B_n\) be sequences in \(\Rset\).
  We define a \emph{real trigonometric polynomial} of degree \(N\) to be
  \[
    \tau_N(t) = \frac{A_0}{2} + \sum_{n=1}^N A_n \cos(n\omega t) + B_n \sin(n\omega t)
  \]
\end{definition}

\begin{lemma}[Orthogonality of the basis functions]
  Let \(m,n \in\Nset_0\)
  \begin{align*}
    \int\limits_0^T \cos(m\omega t)\cos(n\omega t)
    &= \begin{cases}
        T & m = n = 0 \\
        T/2 & m = n > 0 \\
        0 & m \neq n
      \end{cases} \\
    \int\limits_0^T \sin(m\omega t)\sin(n\omega t)
    &= \begin{cases}
        T/2 & m = n \wedge n \neq 0 \\
        0 & m \neq n \\
        0 & m = 0 \vee n = 0
      \end{cases} \\
    \int\limits_0^T \cos(m\omega t)\sin(n\omega t) &= 0
  \end{align*}
\end{lemma}

\begin{definition}
  We denote with \(\Omega\) the space of real valued, \(T\)-periodic, piecewise continuous functions, that have only a finite number of discontinuities, in which both the right and left limit exist, within the interval \([0,T)\).
\end{definition}

\begin{theorem}[Fourier coefficients]
  For any \(f\in\Omega\) we can now define the \emph{Fourier coefficients}
  \begin{align*}
    a_n &= \frac{2}{T}\int\limits_0^T f(t)\cos(n\omega t)\di{t} & a_0 &= \frac{2}{T}\int\limits_0^T f(t)\di{t} \\
    b_n &= \frac{2}{T}\int\limits_0^T f(t)\sin(n\omega t)\di{t} & b_0 &= 0
  \end{align*}
  Worth noting are the special cases when \(n=0\).
\end{theorem}

\begin{definition}[Fourier Polynomial]
  We can now use the Fourier coefficients as sequences for a trigonometric polynomial to obtain a \emph{Fourier Polynomial}
  \[
    S_N(t) = \frac{a_0}{2} + \sum_{n=1}^N a_n\cos(n\omega t) + b_n\sin(n\omega t)
  \]
\end{definition}

\begin{lemma}
  A trigonometric polynomial has the smallest distance (by the \(L^2\) metric) from a function \(f\in\Omega\), iff \(A_n = a_n\) and \(B_n = b_n\), in other words iff it is a Fourier Polynomial.
\end{lemma}

\begin{definition}[Fourier Series]
  We can finally define the \emph{Fourier Series} to be the infinite Fourier Polynomial, by letting \(N\to\infty\)
  \[
    S(t) = \frac{a_0}{2} + \sum_{n=1}^N a_n\cos(n\omega t) + b_n\sin(n\omega t)
  \]
\end{definition}

\begin{theorem}[Fourier coefficients of even and odd functions]
  Recall that a function is said to be \emph{even} if \(f(-x) = f(x)\) or \emph{odd} if \(f(-x) = -f(x)\). We can show that if a function is
  \begin{itemize}
    \item odd, then \(b_n = 0\) for all \(n\), and
      \[
        a_n = \frac{4}{T}\int\limits_0^{T/2} f(t)\cos(n\omega t)\di{t}
      \]
    \item even, then \(a_n = 0\) for all \(n\), and
      \[
        b_n = \frac{4}{T}\int\limits_0^{T/2} f(t)\sin(n\omega t)\di{t}
      \]
  \end{itemize}
\end{theorem}

\begin{lemma}[Linearity of Fourier coefficients]
  Recall that linearity means \(L(\mu x + \lambda y) = \(\mu L(x) + \lambda L(y)\). We then let \(f,g \in\Omega\) be functions with Fourier series and \(h = \mu f + \lambda g\) where \(\mu,\lambda\in\Rset\) are constants.
  By denoting with \(a_n^{(f)}\) the Fourier coefficient \(a_n\) of the function \(f\), and similarly with \(b_n^{(f)}\), it is easily shown that
  \begin{align*}
    a_n^{(h)} &= \mu a_n^{(f)} + \lambda a_n^{(g)} &
    b_n^{(h)} &= \mu b_n^{(f)} + \lambda b_n^{(g)}
  \end{align*}
\end{lemma}

\begin{lemma}[Fourier coefficients after time dilation]
  Let \(f\in\Omega\) be a function with a Fourier Series and \(g(t) = f(rt)\) with \(0 \neq r \in\Rset\). It follows that
  \(a_n^{(g)} = a_n^{(f)}\) and \(b_n^{(g)} = \sgn(r) \cdot b_n^{(f)}\).
\end{lemma}

\begin{lemma}[Fourier coefficients after time translation]
  Let \(f\in\Omega\) be a function with a Fourier Series and \(g(t) = f(t + \tau)\) with \(\tau\in\Rset\). It follows that
  \begin{align*}
    a_n^{(g)} &= \cos(n\omega \tau)\cdot a_n^{(f)} + \sin(n\omega \tau)\cdot b_n^{(f)} & n &\geq 0\\
    b_n^{(g)} &= -\sin(n\omega \tau)\cdot a_n^{(f)} + \cos(n\omega \tau)\cdot b_n^{(f)} & n &> 0
  \end{align*}
\end{lemma}

\begin{theorem}[Fourier theorem]
  For any \(f\in\Omega\) the Fourier series of \(f\) converges in \(L^2\) metric to \(f\).
  \[
    \lim_{N\to\infty} \left\lVert 
      \frac{a_0}{2} + \sum_{n=0}^N a_n \cos(n\omega t) + b_n \sin(n\omega t) - f(t) 
    \right\rVert = 0
  \]
\end{theorem}

\begin{theorem}[Plancherel Parselval theorem]
  Let \(f\in\Omega\) with a Fourier Series with coefficients \(a_n\) and \(b_n\).
  \[
    \frac{a_0}{2} + \sum_{n=1}^\infty \left(a_n^2 + b_n^2\right)
      \leqq \frac{2}{T} \int\limits_0^T \lvert f(t)\rvert^2 \di{t} = \left\lVert f \right\rVert^2
  \]
\end{theorem}

\begin{theorem} Both sequences \(a_n, b_n\) for the Fourier coefficients of a function \(f\in\Omega\) converge to zero.
  \begin{align*}
    \lim_{n\to\infty} a_n 
      &= \lim_{n\to\infty} \frac{2}{T}
          \int\limits_0^T f(t) \cos(n\omega t) \di{t} = 0 \\
    \lim_{n\to\infty} b_n 
      &= \lim_{n\to\infty} \frac{2}{T}
          \int\limits_0^T f(t) \sin(n\omega t) \di{t} = 0
  \end{align*}
\end{theorem}

\begin{theorem}[Rate of convergence of Fourier coefficients]
  If \(f\) is a \(T\)-periodic, \((m-2)\) times differentiable, continuous function. And if its \((m-1)\)-th derivative is pieceweise monotonous and \(\in \Omega\), then there exists a constant \(c \in\Rset\) such that
  \[
    \len{a_n} \leq \frac{c}{n^m} \qquad \len{b_n} \leq \frac{c}{n^m} \qquad m,n\in\Nset
  \]
\end{theorem}

\begin{theorem}[Integration and differentiation of the Fourier series]
  It is possible to show from the previous theorem (and others before) that when \(m\geq 2\) the Fouriers converges \emph{uniformly}. This means that it is possible to integrate or differentiate the series term by term.
  \[
    f'(t) = \sum_{n=1}^\infty b_n n\omega\cos(n\omega t) - a_n n\omega\sin(n\omega t)
  \]
  and
  \begin{align*}
    \int\limits_0^t f(\tau) \di{\tau} &=
      \left(\sum_{n=1}^\infty \frac{b_n}{n\omega} \right)
      + \frac{a_0}{2} t \\
      &+ \left(\sum_{n=1}^\infty
          \frac{a_n}{n\omega}\sin(n\omega t)
          - \frac{b_n}{n\omega}\cos(n\omega t)
        \right)
  \end{align*}
\end{theorem}

\section{Fourier Transform}

\section{License}
\doclicenseThis

\end{document}
% vim: set et ts=2 sw=2 spelllang=en spell linebreak :
