% !TeX program = xelatex
% !TeX encoding = utf8
% !TeX root = komfour_zf.tex

%% TODO: publish to CTAN
\documentclass[]{tex/hsrzf}

%%%%%%%%%%%%%%%%%%%%%%%%%%%%%%%%%%%%%%%%%%%%%%%%%%%
% Packages

%% TODO: publish to CTAN
\usepackage{tex/hsrstud}

%% Language configuration
\usepackage{polyglossia}
\setdefaultlanguage[variant=swiss]{german}

%% Math
\usepackage{amsmath}
\usepackage{amsthm}

%% Layout
\usepackage{multicol}
\usepackage{enumitem}

%% License configuration
\usepackage[
  type={CC},
  modifier={by-nc-sa},
  version={4.0},
  lang={german},
]{doclicense}

%%%%%%%%%%%%%%%%%%%%%%%%%%%%%%%%%%%%%%%%%%%%%%%%%%%
% Metadata

\course{Elektrotechnik}
\module{KomFour}
\semester{Fr\"uhlingssemester 2020}

\authoremail{npross@hsr.ch}
\author{Naoki Pross -- \texttt{\theauthoremail}}

\title{\texttt{\themodule} Zusammenfassung}
\date{\thesemester}

%%%%%%%%%%%%%%%%%%%%%%%%%%%%%%%%%%%%%%%%%%%%%%%%%%%
% Macros and settings

%% Equal by definition
\newcommand\defeq{\overset{\mathrm{def.}}{=}}

%% number sets
\newcommand\Rset{\mathbb{R}}
\newcommand\Cset{\mathbb{C}}

%% Complex operators
\DeclareMathOperator\cjs{cjs}
\newcommand\cjsl[1]{\cos #1 + j\sin #1}

\newcommand\ej[1]{e^{j#1}}
\newcommand\conj[1]{\overline{j #1}}

\renewcommand\Re{Re}
\renewcommand\Im{Im}

%% Theorems
\newtheorem{theorem}{Satz}
\setlist[description]{%
  align=right, labelwidth=2cm, leftmargin=!, %
  format={\normalfont\slshape}}

%%%%%%%%%%%%%%%%%%%%%%%%%%%%%%%%%%%%%%%%%%%%%%%%%%%
% Document

\begin{document}

\maketitle
\tableofcontents

\section{Komplexe Zahlen}
\begin{theorem}[Komplexe Einheit]
\(
  j \defeq +\sqrt{-1} \iff j^2 = -1
\)
\end{theorem}
\begin{theorem}[Multiplikation] \(a, b \in \Cset\) \(\arg a = \phi, \arg b = \theta\)
\begin{description}
  \item[Kartesich] \(a \odot b = (a_1 b_1 - a_2 b_2) + j (a_1 b_2 + a_2 b_1)\)
  \item[Polar] \(a\odot b &= |a|\cdot|b|\exp{j(\phi + \theta)}\)
\end{description} 
\end{theorem}
\begin{theorem}[Division] \(a, b \in \Cset\) \(\arg a = \phi, \arg b = \theta\)
\begin{description}
  \item[Kartesich]
  \item[Polar] \(a / b &= |a|/|b|\exp{j(\phi - \theta)}\)
\end{description} 
\end{theorem}


\subsection{Algebra}
Seien \(a, b \in \Cset\) und \(a = a_1 + ja_2, a_1,a_2 \in \Rset\) und \"ahnlich f\"ur \(b\)
\begin{align*}
  a \oplus b &=  (a_1 + b_1) + j (a_2 + b_2) \\
\end{align*}

\subsection{Neue Operationen}
\subsection{Graphische Darstellung}
\subsubsection{Ebene Geometrie}

\section{Lizenz}
\doclicenseThis

\end{document}
% vim: set et ts=2 sw=2 spelllang=de spell wrap linebreak :
